\documentclass[11pt,a4paper]{article}
\usepackage[utf8]{inputenc}
\usepackage[margin=1in]{geometry}
\usepackage{tcolorbox}
\usepackage{enumitem}
\usepackage{xcolor}
\usepackage{chemfig}
\usepackage{tikz}
\usetikzlibrary{arrows.meta,positioning}

% Define custom colors
\definecolor{headerblue}{RGB}{41,128,185}
\definecolor{lightblue}{RGB}{236,240,245}
\definecolor{sectiongreen}{RGB}{39,174,96}
\definecolor{lightgreen}{RGB}{232,246,239}

% Custom tcolorbox styles
\tcbset{
    mainbox/.style={
        colback=lightblue,
        colframe=headerblue,
        fonttitle=\bfseries\large,
        coltitle=white,
        title filled=true,
        arc=3mm,
        boxrule=1pt
    },
    topicbox/.style={
        colback=lightgreen,
        colframe=sectiongreen,
        fonttitle=\bfseries,
        coltitle=black,
        arc=2mm,
        boxrule=0.8pt,
        left=3mm,
        right=3mm,
        top=2mm,
        bottom=2mm
    }
}

\title{\textbf{FA 25 Exam 4 Study Plan}}
\author{}
\date{}

\begin{document}

\maketitle

\begin{tcolorbox}[mainbox, title=Course Overview]
This study guide covers topics from sections 3.8 through 4.7, including electron configurations, periodic trends, chemical bonding, molecular geometry, and gas laws. Use this document to organize your preparation and track your progress.
\end{tcolorbox}

\section*{Chapter 3: Atomic Structure and Chemical Bonding}

\begin{tcolorbox}[topicbox, title=3.8 Electron Configurations and Diagrams]
\textbf{Key Concepts to Master:}
\begin{itemize}[leftmargin=*]
    \item Understand and apply electron configurations for chemical reactivity
    \item Know the spin quantum number ($m_s$)
    \item Understand the Pauli exclusion principle and sublevels
    \item Depict electron configurations, valence electrons, and core electrons
    \item Identify s, p, d, and f blocks on the periodic table
    \item Use the periodic table to predict electron configurations
    \item Write complete and abbreviated electron configurations
    \item Draw orbital diagrams with proper electron spin notation
\end{itemize}
\end{tcolorbox}

\begin{tcolorbox}[topicbox, title=3.9 Periodic Trends (including Electronegativity)]
\textbf{Key Concepts to Master:}
\begin{itemize}[leftmargin=*]
    \item Predict periodic trends in atomic size
    \item Understand effective nuclear charge ($Z_{eff}$)
    \item Compare metallic character across periods and groups
    \item Predict ionic radii trends
    \item Understand ionization energy trends
    \item Apply electron affinity concepts
    \item Compare electronegativity values
    \item Identify paramagnetic vs. diamagnetic atoms/ions
\end{itemize}
\end{tcolorbox}

\begin{tcolorbox}[topicbox, title=3.10 Ionic Bonding, Lattice Energy, and Lewis Structures]
\textbf{Key Concepts to Master:}
\begin{itemize}[leftmargin=*]
    \item Define ionic, covalent, and metallic bonding
    \item Understand Lewis dot theory
    \item Apply the octet rule
    \item Draw Lewis structures for ionic compounds
    \item Understand lattice energy and its relationship to ion charge and size
\end{itemize}
\end{tcolorbox}

\begin{tcolorbox}[topicbox, title=3.11 Covalent Bonding, Lewis Structures, and Formal Charge]
\textbf{Key Concepts to Master:}
\begin{itemize}[leftmargin=*]
    \item Draw Lewis structures for covalent compounds and polyatomic ions
    \item Understand bond polarity, dipole moment, and partial charge
    \item Define and draw resonance structures
    \item Calculate formal charge for atoms in Lewis structures
    \item Determine magnitude and location of formal charge
    \item Depict single, double, and triple bonds correctly
\end{itemize}
\end{tcolorbox}

\section*{Chapter 4: Molecular Geometry and Gases}

\begin{tcolorbox}[topicbox, title=4.1 VSEPR Theory, Molecular Shapes, and Polarity]
\textbf{Key Concepts to Master:}
\begin{itemize}[leftmargin=*]
    \item Understand VSEPR theory fundamentals
    \item Know the five basic shapes: linear (2), trigonal planar (3), tetrahedral (4), trigonal bipyramidal (5), octahedral (6)
    \item Memorize bond angles for each basic shape
    \item Distinguish between electron geometry and molecular geometry
    \item Understand effects of lone pairs on shape, bond angle, and polarity
    \item Recognize and draw correct molecular geometries
    \item Predict bond angles in polyatomic molecules
\end{itemize}

\vspace{0.5cm}
\textbf{Five Basic Molecular Geometries:}

\begin{center}
\begin{tabular}{cc}
\begin{minipage}{0.45\textwidth}
\centering
\textbf{Linear (2 electron groups)}\\
\textbf{Bond angle: 180°}\\[0.3cm]
\chemfig{X-[::180]A-[:0]X}\\[0.2cm]
Example: CO$_2$, BeH$_2$
\end{minipage}
&
\begin{minipage}{0.45\textwidth}
\centering
\textbf{Trigonal Planar (3 groups)}\\
\textbf{Bond angles: 120°}\\[0.3cm]
\chemfig{X-[::120]A(-[:0]X)-[::240]X}\\[0.2cm]
Example: BF$_3$, SO$_3$
\end{minipage}
\end{tabular}

\vspace{0.5cm}

\begin{tabular}{cc}
\begin{minipage}{0.45\textwidth}
\centering
\textbf{Tetrahedral (4 groups)}\\
\textbf{Bond angles: 109.5°}\\[0.3cm]
\chemfig{X-[:90]A(-[:210]X)(-[:330]X)-[:-90]X}\\[0.2cm]
Example: CH$_4$, NH$_4^+$
\end{minipage}
&
\begin{minipage}{0.45\textwidth}
\centering
\textbf{Trigonal Bipyramidal (5 groups)}\\
\textbf{Angles: 120° (eq), 90° (ax)}\\[0.3cm]
\chemfig{X-[:90]A(-[:210]X)(-[:330]X)(-[:0]X)-[:-90]X}\\[0.2cm]
Example: PCl$_5$, PF$_5$
\end{minipage}
\end{tabular}

\vspace{0.5cm}

\begin{minipage}{\textwidth}
\centering
\textbf{Octahedral (6 electron groups)}\\
\textbf{Bond angles: 90°}\\[0.3cm]
\chemfig{X-[:90]A(-[:180]X)(-[:0]X)(-[:210]X)(-[:330]X)-[:-90]X}\\[0.2cm]
Example: SF$_6$, PF$_6^-$
\end{minipage}
\end{center}

\end{tcolorbox}

\begin{tcolorbox}[topicbox, title=4.2 Valence Bond Theory, Hybridization, and Bonding]
\textbf{Key Concepts to Master:}
\begin{itemize}[leftmargin=*]
    \item Define hybridization and the role of atomic orbitals
    \item Understand common hybridizations: $sp^3$, $sp^2$, and $sp$
    \item Know expanded octet hybridizations: $sp^3d$ and $sp^3d^2$
    \item Recognize hybridization of atoms in polyatomic molecules
    \item Distinguish between sigma ($\sigma$) and pi ($\pi$) bonds
\end{itemize}
\end{tcolorbox}

\begin{tcolorbox}[topicbox, title=4.3 Molecular Polarity, IMF, and Boiling Point]
\textbf{Key Concepts to Master:}
\begin{itemize}[leftmargin=*]
    \item Understand molecular polarity as a function of bond polarity and geometry
    \item Relate physical properties to molecular polarity
    \item Know types of intermolecular forces (IMF): London dispersion, dipole-dipole, hydrogen bonding
    \item Rank physical properties based on molecular formula and IMF strength
\end{itemize}
\end{tcolorbox}

\begin{tcolorbox}[topicbox, title=4.4 Gases and Ideal Gas Laws]
\textbf{Key Concepts to Master:}
\begin{itemize}[leftmargin=*]
    \item Understand and apply Boyle's Law ($P_1V_1 = P_2V_2$)
    \item Understand and apply Charles's Law ($\frac{V_1}{T_1} = \frac{V_2}{T_2}$)
    \item Understand and apply Avogadro's Law
    \item Apply the Ideal Gas Law ($PV = nRT$)
    \item Solve problems using various gas law equations
\end{itemize}
\end{tcolorbox}

\begin{tcolorbox}[topicbox, title=4.5 STP, Molar Volume, and Density of Gases]
\textbf{Key Concepts to Master:}
\begin{itemize}[leftmargin=*]
    \item Define standard temperature and pressure (STP)
    \item Know molar volume of an ideal gas (22.4 L/mol at STP)
    \item Understand relationship between molar volume, molar mass, and density
    \item Calculate gas density from molar mass and vice versa
\end{itemize}
\end{tcolorbox}

\begin{tcolorbox}[topicbox, title=4.6 Dalton's Law and Partial Pressure]
\textbf{Key Concepts to Master:}
\begin{itemize}[leftmargin=*]
    \item Define partial pressure of gaseous components in a mixture
    \item Apply Dalton's Law of Partial Pressures ($P_{total} = P_1 + P_2 + ...$)
    \item Define and calculate mole fraction of components
    \item Interconvert between partial pressure and mole fraction
\end{itemize}
\end{tcolorbox}

\begin{tcolorbox}[topicbox, title=4.7 Effusion and Diffusion]
\textbf{Key Concepts to Master:}
\begin{itemize}[leftmargin=*]
    \item Understand Graham's Law of Effusion
    \item Relate molecular velocity to molar mass and temperature
    \item Interconvert between molar mass, gas density, and average molecular velocity
    \item Compare rates of effusion/diffusion for different gases
\end{itemize}
\end{tcolorbox}

\vspace{0.5cm}

\begin{tcolorbox}[mainbox, title=Study Tips and Schedule]
\textbf{Recommended Approach:}
\begin{enumerate}[leftmargin=*]
    \item Review one topic box per study session (2-3 hours each)
    \item Work through practice problems immediately after reviewing concepts
    \item Create summary sheets for each section with key equations and concepts
    \item Focus extra time on topics involving calculations (gas laws, formal charge)
    \item Practice drawing structures and geometries repeatedly
    \item Form study groups to quiz each other on trends and definitions
    \item Review all material 2-3 days before the exam
\end{enumerate}
\end{tcolorbox}

\end{document}
