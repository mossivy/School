\lecture{2}{Tue 02 Dec}{Exam3Review}
\section{Chapter 11: Introduction to the Nervous System and Nervous Tissue}

\begin{tcolorbox}[reviewstyle, title={What structures are in the CNS versus the PNS?}]
\textbf{My Answer:}
The brain and the spinal cord vs the peripheral nerves and neurons 
\medskip
\textbf{Ideal Answer:}
\medskip
\textbf{\color{red}{Where I Went Wrong:}}


\end{tcolorbox}

\begin{tcolorbox}[reviewstyle, title={  Name the neuroglial cells present in the CNS and the PNS and describe their functions.
}]

\textbf{My Answer:}
Forgot


\medskip
\textbf{Ideal Answer:}

    Astrocytes: These cells anchor neurons and blood vessels, regulate the extracellular environment, facilitate the formation of the blood-brain barrier, and repair damaged tissue.


    Oligodendrocytes: They myelinate certain axons in the CNS, which helps speed up the conduction of action potentials.


    Microglial cells: These act as phagocytes, ingesting and breaking down waste products and pathogens.


    Ependymal cells: They line cavities in the brain and spinal cord, with cilia that circulate cerebrospinal fluid, and some also secrete this fluid.

\medskip
\textbf{\color{red}{Where I Went Wrong:}}
AMOE nmueumonic

\end{tcolorbox}

\begin{tcolorbox}[reviewstyle, title={Draw a neuron and label the parts.}]
\textbf{My Answer:}
		\includegraphics[width=0.75\textwidth]{figures/axon.png}

Also dendrites but whatevs


\medskip
\textbf{Ideal Answer:}



\medskip
\textbf{\color{red}{Where I Went Wrong:}}


\end{tcolorbox}

\begin{tcolorbox}[reviewstyle, title={
Where do you find the different organelles of a neuron? }]
\textbf{My Answer:}
Axon body


\medskip
\textbf{Ideal Answer:}
cell body


\medskip
\textbf{\color{red}{Where I Went Wrong:}}
cell not axon body

\end{tcolorbox}

\begin{tcolorbox}[reviewstyle, title={
  Be able to distinguish between multipolar, bipolar, and pseudounipolar neurons and describe
the location of each.}]
\textbf{My Answer:}

Multipolar -- axons going in multiple directions
in the brain

Psuedounipolar - axons coming out in different direction but all going 1 direction
in the pns

Bipolar - going in 2 directions
forgot
\medskip
\textbf{Ideal Answer:}
Multi: one axon with many dendrites, motor and interneurons

Bipolar: one axon and one dendrite, sensory (olfaction and retina)

Psuedounipolar: 2 axons, sensory pns

\medskip
\textbf{\color{red}{Where I Went Wrong:}}


\end{tcolorbox}

\begin{tcolorbox}[reviewstyle, title={
	describe sensory neurons, interneurons, and motor neurons.}]
\textbf{My Answer:}

sensory --> bring signals to the CNS
interneuron --> process signal in the brain, multipolar
motor --> control some bodily function

\medskip
\textbf{Ideal Answer:}



\medskip
\textbf{\color{red}{Where I Went Wrong:}}


\end{tcolorbox}
\begin{tcolorbox}[reviewstyle, title={What is myelin and why are some axons myelinated?}]
\textbf{My Answer:}

myelin is fat and it is wrapped around the axon to make action potentials go faster

\medskip
\textbf{Ideal Answer:}
also insulation and efficiency


\medskip
\textbf{\color{red}{Where I Went Wrong:}}


\end{tcolorbox}
\begin{tcolorbox}[reviewstyle, title={Describe the different types of ion channels present at different locations in the neuron.}]
\textbf{My Answer:}

Na+ outside and K+ plus insude

Also Ca

\medskip
\textbf{Ideal Answer:}
Ligand gated, Voltage Gated Na K, Voltage gated Ca, Leak


\medskip
\textbf{\color{red}{Where I Went Wrong:}}


\end{tcolorbox}
\begin{tcolorbox}[reviewstyle, title={What is the typical resting membrane potential for a neuron?}]
\textbf{My Answer:}

-70mV

\medskip
\textbf{Ideal Answer:}

duh

\medskip
\textbf{\color{red}{Where I Went Wrong:}}


\end{tcolorbox}
\begin{tcolorbox}[reviewstyle, title={Describe the similarities and differences between local potentials versus action potentials.}]
\textbf{My Answer:}

local potentials do not travel all the way down an axon

but both change in the resting potential

\medskip
\textbf{Ideal Answer:}

got the jist

\medskip
\textbf{\color{red}{Where I Went Wrong:}}


\end{tcolorbox}
\begin{tcolorbox}[reviewstyle, title={Draw an action potential, label the phases, and describe which channels are responsible for each phase.}]
\textbf{My Answer:}

depolarization 
polarization
hyperpolarization
refactory period

\medskip
\textbf{Ideal Answer:}
depolarization: Na open
repolarization: Na close, K open
hyperpolarization: K still open


\medskip
\textbf{\color{red}{Where I Went Wrong:}}


\end{tcolorbox}
\begin{tcolorbox}[reviewstyle, title={Where does the action potential begin, and what triggers the action potential?}]
\textbf{My Answer:}

axon hillock

opening of Na+ channel

\medskip
\textbf{Ideal Answer:}



\medskip
\textbf{\color{red}{Where I Went Wrong:}}


\end{tcolorbox}
\begin{tcolorbox}[reviewstyle, title={Which ion is typically responsible for local potentials (small depolarizations in a small area)?}]
\textbf{My Answer:}

Na+

\medskip
\textbf{Ideal Answer:}



\medskip
\textbf{\color{red}{Where I Went Wrong:}}


\end{tcolorbox}
\begin{tcolorbox}[reviewstyle, title={Why is the refractory period important?}]
\textbf{My Answer:}

determines how frequently a potential can fire

\medskip
\textbf{Ideal Answer:}

prevents overlapping signals
allows unidirectional flow

\medskip
\textbf{\color{red}{Where I Went Wrong:}}


\end{tcolorbox}
\begin{tcolorbox}[reviewstyle, title={What is saltatory conduction, and why do only some neurons exhibit saltatory conduction?}]
\textbf{My Answer:}

Forgot

\medskip
\textbf{Ideal Answer:}

leap from one myelin sheath gap to another

\medskip
\textbf{\color{red}{Where I Went Wrong:}}


\end{tcolorbox}
\begin{tcolorbox}[reviewstyle, title={Draw a typical synapse. Label the pre-synaptic membrane, the post-synaptic membrane, and the synaptic cleft. Where do you find vesicles containing neurotransmitters?}]
\textbf{My Answer:}

in the pre-synaptic cleft

\medskip
\textbf{Ideal Answer:}



\medskip
\textbf{\color{red}{Where I Went Wrong:}}


\end{tcolorbox}
\begin{tcolorbox}[reviewstyle, title={What is the difference between an EPSP and an IPSP? Predict whether activation of a given neurotransmitter receptor will generate an EPSP or an IPSP.}]
\textbf{My Answer:}

what

\medskip
\textbf{Ideal Answer:}

excitatory postsynaptic potential) and an IPSP (inhibitory postsynaptic potential)
excitatory --> moves close to action potential
typically caused by the opening of ligand-gated sodium or calcium ion channels

inhibitory --> moves away from action potential
usually the result of opening ligand-gated potassium or chloride ion channels.

\medskip
\textbf{\color{red}{Where I Went Wrong:}}


\end{tcolorbox}
\begin{tcolorbox}[reviewstyle, title={why do we need temporal and spatial summation?}]
\textbf{My Answer:}

different ways to create/start an action potention

\medskip
\textbf{Ideal Answer:}

yep pretty much

\medskip
\textbf{\color{red}{Where I Went Wrong:}}


\end{tcolorbox}
\begin{tcolorbox}[reviewstyle, title={What neurotransmitter is present at the neuromuscular junction?}]
\textbf{My Answer:}

acetylcholine

\medskip
\textbf{Ideal Answer:}

yep

\medskip
\textbf{\color{red}{Where I Went Wrong:}}


\end{tcolorbox}
\begin{tcolorbox}[reviewstyle, title={Name two excitatory and two inhibitory neurotransmitters.}]
\textbf{My Answer:}

forgot

\medskip
\textbf{Ideal Answer:}
excite: glutamate, acetylcholine
inhibit: GABA, glyecin


\medskip
\textbf{\color{red}{Where I Went Wrong:}}


\end{tcolorbox}
\begin{tcolorbox}[reviewstyle, title={How do we terminate the action of neurotransmitters? (Name three things that can happen to the neurotransmitter.)}]
\textbf{My Answer:}

bind to g-protein
diffuse in the cells
forgot


\medskip
\textbf{Ideal Answer:}

diffusion, absorption, degredation, reuptake in pre-synaptic cleft

\medskip
\textbf{\color{red}{Where I Went Wrong:}}


\end{tcolorbox}
\begin{tcolorbox}[reviewstyle, title={Describe how interneurons are organized into neuronal pools. Contrast diverging circuits versus converging circuits.}]
\textbf{My Answer:}

diverging circuit on neuron to many
converging circuit is many neuron to less

\medskip
\textbf{Ideal Answer:}
yep


\medskip
\textbf{\color{red}{Where I Went Wrong:}}


\end{tcolorbox}

\section{Chapter 12: The Central Nervous System}

\begin{tcolorbox}[reviewstyle, title={Draw the shape of the brain, label the regions, and list them in order from the medulla oblongata to the cerebral cortex.}]
\textbf{My Answer:}

medulla oblongata
pons
midbrain
thalamus
cingulate gyrus
cerebral cortex
\medskip
\textbf{Ideal Answer:}
    Medulla Oblongata: Located at the base of the brainstem, it regulates autonomic functions.
    Pons: Above the medulla, it regulates breathing and the sleep/wake cycle.
    Midbrain: The top part of the brainstem, it processes visual and auditory stimuli.
    Cerebellum: Posterior to the brainstem, it coordinates voluntary movements.
    Diencephalon: Contains the thalamus and hypothalamus, involved in sensory and autonomic functions.
    Cerebral Cortex: The outermost layer of the cerebrum, responsible for higher brain functions like thought and action.


\medskip
\textbf{\color{red}{Where I Went Wrong:}}


\end{tcolorbox}
\begin{tcolorbox}[reviewstyle, title={Describe the placement of gray and white matter in the brain and its functional significance. Describe how neurons are organized within gray and white matter.}]
\textbf{My Answer:}

gray is at the outside, bodies
white is inner, axons
\medskip
\textbf{Ideal Answer:}



\medskip
\textbf{\color{red}{Where I Went Wrong:}}


\end{tcolorbox}
\begin{tcolorbox}[reviewstyle, title={Name the three types of white matter tracts in the brain and describe their function.}]
\textbf{My Answer:}

idk

\medskip
\textbf{Ideal Answer:}

commissural: connect the right and left hemispheres, corpus callosum
projection: connect axons in same hemisphere, and CNS with lower CNS
association: same hemisphere

\medskip
\textbf{\color{red}{Where I Went Wrong:}}


\end{tcolorbox}
\begin{tcolorbox}[reviewstyle, title={Name the lobes of the cerebrum and describe the functional areas of each.}]
\textbf{My Answer:}

frontal: complex thought
parietal: motor, sensory
temporal: hearing 
insula: tastes
occipital: vision

\medskip
\textbf{Ideal Answer:}



\medskip
\textbf{\color{red}{Where I Went Wrong:}}


\end{tcolorbox}
\begin{tcolorbox}[reviewstyle, title={What is an association area?}]
\textbf{My Answer:}

area of the brain connected to a particular function

\medskip
\textbf{Ideal Answer:}
integrate various types of information. They are crucial for cognitive functions such as processing complex stimuli, 


    Parietal Association Cortex: Handles spatial awareness and attention, enabling recognition of object positions and movements.

    Temporal Association Cortex: Specializes in recognizing complex stimuli, such as faces, and is involved in identifying stimuli.

    Prefrontal Cortex: Involved in behavior modulation, personality, learning, and memory.



\medskip
\textbf{\color{red}{Where I Went Wrong:}}


\end{tcolorbox}
\begin{tcolorbox}[reviewstyle, title={What is the general function of the basal nuclei?}]
\textbf{My Answer:}

to stop signals from erronously being carried

\medskip
\textbf{Ideal Answer:}

this but mainly for movement

\medskip
\textbf{\color{red}{Where I Went Wrong:}}


\end{tcolorbox}
\begin{tcolorbox}[reviewstyle, title={What are the structures of the limbic system? What are the functions of the limbic system?}]
\textbf{My Answer:}

amygdala, thalamus

\medskip
\textbf{Ideal Answer:}
hippocampus and amygdala
involved in memory, learning, emotion, and behavior

\medskip
\textbf{\color{red}{Where I Went Wrong:}}


\end{tcolorbox}
\begin{tcolorbox}[reviewstyle, title={Describe how Broca’s area and Wernicke’s area are involved in speech and language.}]
\textbf{My Answer:}

broca: is speech processing 
wernicke: is motor language

\medskip
\textbf{Ideal Answer:}

broca: language production
wernicke: language comprehension

\medskip
\textbf{\color{red}{Where I Went Wrong:}}


\end{tcolorbox}
\begin{tcolorbox}[reviewstyle, title={Describe the locations of the thalamus and hypothalamus. Describe the functions of the thalamus and hypothalamus.}]
\textbf{My Answer:}

in the middle of the brain
thalamus, the multiplexor of the brain
hypothalamus: same thing but mainly for endocrine system

\medskip
\textbf{Ideal Answer:}



\medskip
\textbf{\color{red}{Where I Went Wrong:}}


\end{tcolorbox}
\begin{tcolorbox}[reviewstyle, title={Where is the cerebellum located? How do axons travel to or from the cerebellum? What is the function of the cerebellum?}]
\textbf{My Answer:}

back of the brain, have to detour, refine motor control

\medskip
\textbf{Ideal Answer:}



\medskip
\textbf{\color{red}{Where I Went Wrong:}}


\end{tcolorbox}
\begin{tcolorbox}[reviewstyle, title={What three structures form the brainstem? Where do you find the reticular formation? What is the function of the reticular formation?}]
\textbf{My Answer:}

medulla oblongata, pons, midbrain 
reticular formation is the center of all of them
idk what the function is 

\medskip
\textbf{Ideal Answer:}
sleep/arousal
autonomic function
homeostasis


\medskip
\textbf{\color{red}{Where I Went Wrong:}}


\end{tcolorbox}
\begin{tcolorbox}[reviewstyle, title={What two structures of the central nervous system are key for maintaining homeostasis?}]
\textbf{My Answer:}

brainstem and Diencephalon

\medskip
\textbf{Ideal Answer:}
sure, and ANS


\medskip
\textbf{\color{red}{Where I Went Wrong:}}


\end{tcolorbox}
\begin{tcolorbox}[reviewstyle, title={How do we measure brain activity?}]
\textbf{My Answer:}

voltage

\medskip
\textbf{Ideal Answer:}

MRI
EEG
PET

\medskip
\textbf{\color{red}{Where I Went Wrong:}}


\end{tcolorbox}
\begin{tcolorbox}[reviewstyle, title={Name the 3 layers of the meninges in order.}]
\textbf{My Answer:}

dura mater 
arachnoid mater
pia mater

\medskip
\textbf{Ideal Answer:}
yep


\medskip
\textbf{\color{red}{Where I Went Wrong:}}


\end{tcolorbox}
\begin{tcolorbox}[reviewstyle, title={Describe the production of CSF.}]
\textbf{My Answer:}

idk

\medskip
\textbf{Ideal Answer:}
t is produced in the brain's ventricles by the choroid plexus, a cluster of capillaries and ependymal cells


\medskip
\textbf{\color{red}{Where I Went Wrong:}}


\end{tcolorbox}
\begin{tcolorbox}[reviewstyle, title={Describe the blood brain barrier.}]
\textbf{My Answer:}

layer around the brain where gases can pass but not blood

\medskip
\textbf{Ideal Answer:}

Endothelial cells and Astrocytes

\includegraphics[width=0.95\textwidth]{figures/bbb.jpg}

\medskip
\textbf{\color{red}{Where I Went Wrong:}}


\end{tcolorbox}
\begin{tcolorbox}[reviewstyle, title={What are the functions of the spinal cord?}]
\textbf{My Answer:}

Control ANS and connect CNS with PNS

\medskip
\textbf{Ideal Answer:}



\medskip
\textbf{\color{red}{Where I Went Wrong:}}


\end{tcolorbox}
\begin{tcolorbox}[reviewstyle, title={What substance do you find in the epidural space? What substance do you find in the subarachnoid space?}]
\textbf{My Answer:}

csf

\medskip
\textbf{Ideal Answer:}

epidural: fat, veins
subarachnoid: CSF

\medskip
\textbf{\color{red}{Where I Went Wrong:}}


\end{tcolorbox}
\begin{tcolorbox}[reviewstyle, title={Describe the anatomy of the spinal cord itself (superior to inferior segments and layers).}]
\textbf{My Answer:}

idk

\medskip
\textbf{Ideal Answer:}

The spinal cord begins as an extension of the brainstem at the foramen magnum and ends at the conus medullaris, located between the first and second lumbar vertebrae. It features two enlargements: the cervical enlargement and the lumbar enlargement, which accommodate the nerves serving the upper and lower limbs, respectively.

dura, arachnoid, pia,


\medskip
\textbf{\color{red}{Where I Went Wrong:}}


\end{tcolorbox}
\begin{tcolorbox}[reviewstyle, title={Describe the organization of white matter and gray matter within the spinal cord.}]
\textbf{My Answer:}

grey is middle
white is outer

\medskip
\textbf{Ideal Answer:}



\medskip
\textbf{\color{red}{Where I Went Wrong:}}


\end{tcolorbox}
\begin{tcolorbox}[reviewstyle, title={What senses are classified as general somatic senses?}]
\textbf{My Answer:}

touch

\medskip
\textbf{Ideal Answer:}

touch
pain
temprature
stretch 
proprioception

\medskip
\textbf{\color{red}{Where I Went Wrong:}}


\end{tcolorbox}
\begin{tcolorbox}[reviewstyle, title={Describe the pathways travelled by the posterior columns pathway and the spinothalamic tract pathway.}]
\textbf{My Answer:}



\medskip
\textbf{Ideal Answer:}
Posterior: sensory first order enter posterior horn, travel through the fasciculus gracilis, synapse in the medulla, cross over in the medulla, ascend to pons and midbrain, synapse in the thalamus, third order to the cortex 

spinothalamic: first-order pain temp and nondisc touch, synapse in post horn, decussation in spinal cord, ascend, synapse in thalamus, 


\medskip
\textbf{\color{red}{Where I Went Wrong:}}


\end{tcolorbox}
\begin{tcolorbox}[reviewstyle, title={Which sensory pathways go through the thalamus?}]
\textbf{My Answer:}

all except olfaction

\medskip
\textbf{Ideal Answer:}

yep

\medskip
\textbf{\color{red}{Where I Went Wrong:}}


\end{tcolorbox}
\begin{tcolorbox}[reviewstyle, title={Describe the corticospinal pathway, naming the neurons involved.}]
\textbf{My Answer:}



\medskip
\textbf{Ideal Answer:}

voluntary movement: 
upper motor, go through midbrain, decussate, spinal cord, synapse with interneuron and the synapse with and lower motor neuron

\medskip
\textbf{\color{red}{Where I Went Wrong:}}


\end{tcolorbox}
\begin{tcolorbox}[reviewstyle, title={What is a “motor program”?}]
\textbf{My Answer:}

lots of signals together to make a movement

\medskip
\textbf{Ideal Answer:}



\medskip
\textbf{\color{red}{Where I Went Wrong:}}


\end{tcolorbox}
\begin{tcolorbox}[reviewstyle, title={How does the cerebellum reduce motor error?}]
\textbf{My Answer:}



\medskip
\textbf{Ideal Answer:}

receives sensory info and compares to intended movement and finds discrepency
learns and adapts

\medskip
\textbf{\color{red}{Where I Went Wrong:}}


\end{tcolorbox}

\section{Chapter 13: The Peripheral Nervous System}

\begin{tcolorbox}[reviewstyle, title={Define the following terms within the context of the peripheral nervous system: sensory / motor; somatic / visceral; afferent / efferent.}]
\textbf{My Answer:}

know this

\medskip
\textbf{Ideal Answer:}



\medskip
\textbf{\color{red}{Where I Went Wrong:}}


\end{tcolorbox}
\begin{tcolorbox}[reviewstyle, title={Describe the type of information carried in each division of the peripheral nervous system.}]
\textbf{My Answer:}

skip

\medskip
\textbf{Ideal Answer:}



\medskip
\textbf{\color{red}{Where I Went Wrong:}}


\end{tcolorbox}
\begin{tcolorbox}[reviewstyle, title={Describe the structure of a nerve (the layers and coverings).}]
\textbf{My Answer:}

skip

\medskip
\textbf{Ideal Answer:}



\medskip
\textbf{\color{red}{Where I Went Wrong:}}


\end{tcolorbox}
\begin{tcolorbox}[reviewstyle, title={Name the 12 pairs of cranial nerves. Explain the physiological significance of each pair. Which cranial nerves carry parasympathetic innervation?}]
\textbf{My Answer:}

1. Olfactory

2. Ocular

3. Oculomotor

4. Trigeminal

5. Facial

6. 

7. 

8. Vestibulocochlear 

9. Glossopharyngeal

10. Vagus

11. 

12. Accesory

\medskip
\textbf{Ideal Answer:}

1. Olfactory - smell

2. Optic - vision 

3. Oculomotor - eye movement  

4. Trochlear - eye movement  

5. Trigeminal - facial sensation and chewing 

6. Abducens - lateral eye movement 

7. Facial - taste, facial expression, salivary glands 

8. Vestibulocochlear - hearing and  balance 

9. Glossopharyngeal - swallow, taste

10. Vagus - parasympathetic, internal organs

11. Accesory - shoulder and neck muscles

12. Hypoglossal - tongue movement

\medskip
\textbf{\color{red}{Where I Went Wrong:}}


\end{tcolorbox}
\begin{tcolorbox}[reviewstyle, title={What is the difference between an anterior root and a posterior root?}]
\textbf{My Answer:}

anterior, efferent, post is afferent

\medskip
\textbf{Ideal Answer:}



\medskip
\textbf{\color{red}{Where I Went Wrong:}}


\end{tcolorbox}
\begin{tcolorbox}[reviewstyle, title={What is the difference between an anterior root and an anterior ramus?}]
\textbf{My Answer:}



\medskip
\textbf{Ideal Answer:}

anterior root goes to the spinal cord whereas ramus goes to the peripheral

\medskip
\textbf{\color{red}{Where I Went Wrong:}}


\end{tcolorbox}
\begin{tcolorbox}[reviewstyle, title={What is a nerve plexus? Name the 4 nerve plexuses. Name the nerves from the (lab) tag list that come off each plexus and describe what they innervate.}]
\textbf{My Answer:}

network of nerves, cervical, lumbar, thoracic, 

\medskip
\textbf{Ideal Answer:}
cervical, brachial, lumbar, sacral

\medskip
\textbf{\color{red}{Where I Went Wrong:}}


\end{tcolorbox}
\begin{tcolorbox}[reviewstyle, title={How do sensory neurons collect sensory input? What is a “receptor”? Describe the types of receptors involved in sensation and how those receptors are activated.}]
\textbf{My Answer:}



\medskip
\textbf{Ideal Answer:}



\medskip
\textbf{\color{red}{Where I Went Wrong:}}


\end{tcolorbox}
\begin{tcolorbox}[reviewstyle, title={What is a receptive field and how do you measure the size of a receptive field?}]
\textbf{My Answer:}

how close a sensory is discriminatory

\medskip
\textbf{Ideal Answer:}



\medskip
\textbf{\color{red}{Where I Went Wrong:}}


\end{tcolorbox}
\begin{tcolorbox}[reviewstyle, title={What is a dermatome?}]
\textbf{My Answer:}

where touch 

\medskip
\textbf{Ideal Answer:}

A dermatome is a segment of skin supplied by a specific spinal nerve

\medskip
\textbf{\color{red}{Where I Went Wrong:}}


\end{tcolorbox}
\begin{tcolorbox}[reviewstyle, title={Why do people sometimes experience referred pain?}]
\textbf{My Answer:}

idk

\medskip
\textbf{Ideal Answer:}

one nerve can innvervate multiple things. ie: heart attack as skin pain

\medskip
\textbf{\color{red}{Where I Went Wrong:}}


\end{tcolorbox}
\begin{tcolorbox}[reviewstyle, title={Describe how neurons connect to form a reflex arc.}]
\textbf{My Answer:}

skip

\medskip
\textbf{Ideal Answer:}



\medskip
\textbf{\color{red}{Where I Went Wrong:}}


\end{tcolorbox}
\begin{tcolorbox}[reviewstyle, title={What sensory input is collected by the muscle spindles and the Golgi tendon organ?}]
\textbf{My Answer:}

tension

\medskip
\textbf{Ideal Answer:}



\medskip
\textbf{\color{red}{Where I Went Wrong:}}


\end{tcolorbox}
\begin{tcolorbox}[reviewstyle, title={Describe a simple stretch reflex; a flexion/withdrawal reflex; a crossed-extension reflex; a gag reflex; the corneal blink reflex.}]
\textbf{My Answer:}

skip

\medskip
\textbf{Ideal Answer:}



\medskip
\textbf{\color{red}{Where I Went Wrong:}}


\end{tcolorbox}

\section{Chapter 14: The Autonomic Nervous System and Homeostasis}

\begin{tcolorbox}[reviewstyle, title={Name the divisions of the PNS and describe the type of information found in each.}]
\textbf{My Answer:}



\medskip
\textbf{Ideal Answer:}



\medskip
\textbf{\color{red}{Where I Went Wrong:}}


\end{tcolorbox}
\begin{tcolorbox}[reviewstyle, title={Which division of the peripheral nervous system is also known as the autonomic nervous system?}]
\textbf{My Answer:}

visceral motor

\medskip
\textbf{Ideal Answer:}



\medskip
\textbf{\color{red}{Where I Went Wrong:}}


\end{tcolorbox}
\begin{tcolorbox}[reviewstyle, title={In which regions do you find the outflow pathways for the sympathetic and parasympathetic nervous systems?}]
\textbf{My Answer:}



\medskip
\textbf{Ideal Answer:}
sympathetic out: thoracic and lumbar
para: brainstem and sacral



\medskip
\textbf{\color{red}{Where I Went Wrong:}}


\end{tcolorbox}
\begin{tcolorbox}[reviewstyle, title={Describe the locations for the peripheral ganglia for the sympathetic and parasympathetic nervous systems.}]
\textbf{My Answer:}
\medskip
\textbf{Ideal Answer:}

symp chain ganglia 
para: near target organ




\medskip
\textbf{\color{red}{Where I Went Wrong:}}


\end{tcolorbox}
\begin{tcolorbox}[reviewstyle, title={Compare and contrast the function of the sympathetic versus parasympathetic nervous systems.}]
\textbf{My Answer:}

skip

\medskip
\textbf{Ideal Answer:}



\medskip
\textbf{\color{red}{Where I Went Wrong:}}


\end{tcolorbox}
\begin{tcolorbox}[reviewstyle, title={Explain the concept of dual innervation.}]
\textbf{My Answer:}

1 thing/organ being innervated twice

\medskip
\textbf{Ideal Answer:}
innvervated by sympathetic and parasympathetic nerves


\medskip
\textbf{\color{red}{Where I Went Wrong:}}


\end{tcolorbox}
\begin{tcolorbox}[reviewstyle, title={Predict the effect of the sympathetic or a parasympathetic nervous system on the following structures:
• Pupil
• Heart
• Lungs
• Digestive organs
• Urinary organs
• Digestive and urinary sphincters}]
\textbf{My Answer:}

Skip

\medskip
\textbf{Ideal Answer:}



\medskip
\textbf{\color{red}{Where I Went Wrong:}}


\end{tcolorbox}
\begin{tcolorbox}[reviewstyle, title={Describe the effect of the sympathetic nervous system on various different blood vessels.}]
\textbf{My Answer:}
skip


\medskip
\textbf{Ideal Answer:}



\medskip
\textbf{\color{red}{Where I Went Wrong:}}


\end{tcolorbox}
\begin{tcolorbox}[reviewstyle, title={What is the role of the adrenal medulla, and how is the adrenal medulla activated?}]
\textbf{My Answer:}

adrenalin, sympathetic nervous system

\medskip
\textbf{Ideal Answer:}



\medskip
\textbf{\color{red}{Where I Went Wrong:}}


\end{tcolorbox}
\begin{tcolorbox}[reviewstyle, title={Which neurotransmitter is the key neurotransmitter for the:
• Sympathetic and parasympathetic ganglia
• Sympathetic post-ganglionic neurons
• Parasympathetic post-ganglionic neurons}]
\textbf{My Answer:}

idk

\medskip
\textbf{Ideal Answer:}
both: acetylcholine, binds to nicotinic receptors
sympathetic: norepinephrine, epinephrine
para: AcH


\medskip
\textbf{\color{red}{Where I Went Wrong:}}


\end{tcolorbox}
\begin{tcolorbox}[reviewstyle, title={What types of receptors are activated by epinephrine and norepinephrine?}]
\textbf{My Answer:}

nicotinic

\medskip
\textbf{Ideal Answer:}

Adrenergic Receptors: alpha and beta

\medskip
\textbf{\color{red}{Where I Went Wrong:}}


\end{tcolorbox}
\begin{tcolorbox}[reviewstyle, title={What types of receptors are activated by acetylcholine?}]
\textbf{My Answer:}

nicotinic

\medskip
\textbf{Ideal Answer:}
nicotinic, muscarinic


\medskip
\textbf{\color{red}{Where I Went Wrong:}}


\end{tcolorbox}
\begin{tcolorbox}[reviewstyle, title={Explain the basic mechanisms of action for Botox and Nicotine.}]
\textbf{My Answer:}



\medskip
\textbf{Ideal Answer:}
Botox binds to NMJ and blocks ACH release
Nicotine: mimics ach


\medskip
\textbf{\color{red}{Where I Went Wrong:}}


\end{tcolorbox}
\begin{tcolorbox}[reviewstyle, title={How does the hypothalamus direct the autonomic nervous system?}]
\textbf{My Answer:}

skip

\medskip
\textbf{Ideal Answer:}



\medskip
\textbf{\color{red}{Where I Went Wrong:}}


\end{tcolorbox}
\begin{tcolorbox}[reviewstyle, title={What is autonomic tone?}]
\textbf{My Answer:}

always on ans

\medskip
\textbf{Ideal Answer:}



\medskip
\textbf{\color{red}{Where I Went Wrong:}}


\end{tcolorbox}
\begin{tcolorbox}[reviewstyle, title={Where are the autonomic centers that control the sympathetic and parasympathetic neurons?}]
\textbf{My Answer:}

hypothalamus, brainstep

\medskip
\textbf{Ideal Answer:}
brainstem


\medskip
\textbf{\color{red}{Where I Went Wrong:}}


\end{tcolorbox}

\section{Chapter 15: The Special Senses}

\begin{tcolorbox}[reviewstyle, title={Special senses: how are special senses the same and different from general senses? Give examples of each.}]
\textbf{My Answer:}

skip

\medskip
\textbf{Ideal Answer:}



\medskip
\textbf{\color{red}{Where I Went Wrong:}}


\end{tcolorbox}
\begin{tcolorbox}[reviewstyle, title={Describe transduction.}]
\textbf{My Answer:}

turn something into signal

\medskip
\textbf{Ideal Answer:}



\medskip
\textbf{\color{red}{Where I Went Wrong:}}


\end{tcolorbox}
\begin{tcolorbox}[reviewstyle, title={For each special sense, describe the following:
• The sense organ that collects that stimulus information
• The type of receptor that is activated
• Which cranial nerve carries the sensation
• Where the primary sensory cortex is located}]
\textbf{My Answer:}

skip

\medskip
\textbf{Ideal Answer:}



\medskip
\textbf{\color{red}{Where I Went Wrong:}}


\end{tcolorbox}
\begin{tcolorbox}[reviewstyle, title={What smells can you sense? How and why? How do smells trigger memories?}]
\textbf{My Answer:}



\medskip
\textbf{Ideal Answer:}
odors, linked to limbic system


\medskip
\textbf{\color{red}{Where I Went Wrong:}}


\end{tcolorbox}
\begin{tcolorbox}[reviewstyle, title={What tastes can you sense? How are taste receptors activated?}]
\textbf{My Answer:}

skip

\medskip
\textbf{Ideal Answer:}



\medskip
\textbf{\color{red}{Where I Went Wrong:}}


\end{tcolorbox}
\begin{tcolorbox}[reviewstyle, title={Describe the pathway that light takes as it enters your eye.}]
\textbf{My Answer:}



\medskip
\textbf{Ideal Answer:}
cornea, aqueous humor, pupil, lens, vitreous humor, retina, optic nerve


\medskip
\textbf{\color{red}{Where I Went Wrong:}}


\end{tcolorbox}
\begin{tcolorbox}[reviewstyle, title={Describe the three layers of the eyeball.}]
\textbf{My Answer:}

fibrous, vascular, nervous 

\medskip
\textbf{Ideal Answer:}



\medskip
\textbf{\color{red}{Where I Went Wrong:}}


\end{tcolorbox}
\begin{tcolorbox}[reviewstyle, title={Why do you have a blind spot?}]
\textbf{My Answer:}

spot without cones and rods

\medskip
\textbf{Ideal Answer:}



\medskip
\textbf{\color{red}{Where I Went Wrong:}}


\end{tcolorbox}
\begin{tcolorbox}[reviewstyle, title={How do the cornea, the lens, and the ciliary body work together to focus the image on the retina?}]
\textbf{My Answer:}

nah

\medskip
\textbf{Ideal Answer:}



\medskip
\textbf{\color{red}{Where I Went Wrong:}}


\end{tcolorbox}
\begin{tcolorbox}[reviewstyle, title={Why do some people need corrective lenses for clear vision?}]
\textbf{My Answer:}

focal point

\medskip
\textbf{Ideal Answer:}



\medskip
\textbf{\color{red}{Where I Went Wrong:}}


\end{tcolorbox}
\begin{tcolorbox}[reviewstyle, title={How do extrinsic eye muscles orient the eyeball?}]
\textbf{My Answer:}

skip

\medskip
\textbf{Ideal Answer:}



\medskip
\textbf{\color{red}{Where I Went Wrong:}}


\end{tcolorbox}
\begin{tcolorbox}[reviewstyle, title={How does the autonomic nervous system control pupil size?}]
\textbf{My Answer:}

light and such

\medskip
\textbf{Ideal Answer:}



\medskip
\textbf{\color{red}{Where I Went Wrong:}}


\end{tcolorbox}
\begin{tcolorbox}[reviewstyle, title={Why do half of the optic nerve fibers cross at the optic chiasm?}]
\textbf{My Answer:}

to have division of left and right sight

\medskip
\textbf{Ideal Answer:}



\medskip
\textbf{\color{red}{Where I Went Wrong:}}


\end{tcolorbox}
\begin{tcolorbox}[reviewstyle, title={Compare and contrast cone and rod photoreceptors.}]
\textbf{My Answer:}

cone color
rod black and white, more

\medskip
\textbf{Ideal Answer:}



\medskip
\textbf{\color{red}{Where I Went Wrong:}}


\end{tcolorbox}
\begin{tcolorbox}[reviewstyle, title={Describe the structures of the outer, middle, and inner ear.}]
\textbf{My Answer:}

skip

\medskip
\textbf{Ideal Answer:}



\medskip
\textbf{\color{red}{Where I Went Wrong:}}


\end{tcolorbox}
\begin{tcolorbox}[reviewstyle, title={Which bone houses the structures of the inner ear?}]
\textbf{My Answer:}

temporal

\medskip
\textbf{Ideal Answer:}



\medskip
\textbf{\color{red}{Where I Went Wrong:}}


\end{tcolorbox}
\begin{tcolorbox}[reviewstyle, title={Describe the structures that conduct auditory signals from the air to the hair cells of the cochlea.}]
\textbf{My Answer:}

skip

\medskip
\textbf{Ideal Answer:}



\medskip
\textbf{\color{red}{Where I Went Wrong:}}


\end{tcolorbox}
\begin{tcolorbox}[reviewstyle, title={How are the vibrations of sound transduced into electrical signals?}]
\textbf{My Answer:}

skip

\medskip
\textbf{Ideal Answer:}



\medskip
\textbf{\color{red}{Where I Went Wrong:}}


\end{tcolorbox}
\begin{tcolorbox}[reviewstyle, title={Describe how the utricle and saccule detect head tilt and linear acceleration.}]
\textbf{My Answer:}



\medskip
\textbf{Ideal Answer:}



\medskip
\textbf{\color{red}{Where I Went Wrong:}}


\end{tcolorbox}
\begin{tcolorbox}[reviewstyle, title={Describe how the semicircular ducts detect rotational movements.}]
\textbf{My Answer:}



\medskip
\textbf{Ideal Answer:}



\medskip
\textbf{\color{red}{Where I Went Wrong:}}


\end{tcolorbox}
\begin{tcolorbox}[reviewstyle, title={Which brain structures collect vestibular sensations in order to perform their functions?}]
\textbf{My Answer:}



\medskip
\textbf{Ideal Answer:}

thalamus and parietal lobe

\medskip
\textbf{\color{red}{Where I Went Wrong:}}


\end{tcolorbox}
\begin{tcolorbox}[reviewstyle, title={What roles do the frontal lobes and the limbic system play in sensation?}]
\textbf{My Answer:}

process and interpret

\medskip
\textbf{Ideal Answer:}
higher level processing


\medskip
\textbf{\color{red}{Where I Went Wrong:}}


\end{tcolorbox}
